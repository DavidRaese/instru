\newpage

\section{ICP-OES}
\subsection{Experimenteller Teil}

\subsubsection{Geräteeinstellungen}
Für die Messungen wurde ein \textit{Optima 3000 XL} verwendet. Alle Geräteeinstellungen sind in
Tabelle \ref{tab:conditions} aufgeführt. Die verwendeten Wellenlängen zur Vermessung sind Ca (317.933nm),	K (766.490nm),	Mg (285.213nm),	Na (589.592nm),	Fe (239.562nm), Mn (257.610nm). Bei diesen Wellenlängen das Signal zu Rauschen Verhältins am geringsten ist.
% Table generated by Excel2LaTeX from sheet 'ICP-OES Ergebnis'
\begin{table}[htbp]
  \centering
  \caption{Einstellungen des Geräts}
    \begin{tabular}{ll}
    \toprule
    Rf Generator Freequency & 40 MHz \\
    Rf Power & 1300 W \\
    Plasma Argon Flow & 15 L/min \\
    Auxiliary Argon Flow & 0,5 L/min \\
    Plasma Viewing Mode & Axial \\
    Spraychamber & Cyclonic \\
    Read Delay & 18 sec \\
    Autosampler Rate & 1,5 ml/min \\
    Optical System & Echelle \\
    \bottomrule
    \end{tabular}%
  \label{tab:conditions}%
\end{table}%

% Ca 317.933	K 766.490	Mg 285.213	Na 589.592	Fe 239.562	Mn 257.610


\subsubsection{Verwendete Reagenzien}
Für die Kalibration und Wiederfindungsexperimente wurden Einzelelement Stocklösungen ($1 \frac{g}{L}$) von der Firma Roth
verwendet. Als Referenzproben dienten inhouse vermessene Joghurts, die gemeinsam mit
den Eigenproben aufgeschlossen wurden. Verdünnt wurden alle Analyten mittels einer 1\% \ch{HNO3}
Lösung. Alle verwendeten Maßkolben wurden mittels deionisiertem Wasser gereinigt um Kreuzkontaminationen 
zu vermeiden.

\subsubsection{Kalibration}
Zur Bestimmung der Konzentration wurde eine externe Kalibration mittels der Einzelelement Stocklösungen
im Konzentrationsbereich $0,2$ - $5 \frac{mg}{L}$ für Mengen- und $0,02$ - $0,5 \frac{mg}{L}$ für
Nebenelemente angefertigt. Um einen eventuellen Gerätedrift auszugleichen wurden alle Proben noch zusätzlich mit
$1 \frac{mg}{L}$ \ch{Sc} versetzt.

\subsubsection{Qualitätssicherung}
Um die Linearität der Eichgraden zu gewährleisten wurden die Korrelationskoeffizienten ($R^2$)
bestimmt. Zur Evaluierung der Sensitivität der Methode wurde die Nachweißgrenze (LOD),
sowie die Bestimmungsgrenze (LOQ) für die einzelnen Elemente ermittelt (Tabelle 2). Abschließend
wurde die Qualitätssicherung noch mittels Wiederfindungsexperimenten der jeweiligen Elemente überprüft.
Dafür wurden zwei Referenzproben mit $1 \frac{mg}{L}$ für die Mengen- und $0,1 \frac{mg}{L}$
für die Nebenelemente gespicked.
\begin{table}[htbp]
  \centering
  \caption{Methodenvalidierung}
    \begin{tabular}{lcccc}
    \toprule
    Element & R$^2$ & LOD $\frac{mg}{kg}$ & LOQ $\frac{mg}{kg}$ & WFR [\%] \\
    \midrule
    Ca    & 1,00  & 0,00  & 0,11  & 84 \\
    K     & 1,00  & 0,00  & 0,00  & 113 \\
    Mg    & 1,00  & 0,36  & 0,84  & 101 \\
    Na    & 1,00  & 0,01  & 0,02  & 98 \\
    Fe    & 1,00  & 0,00  & 0,00  & 95 \\
    Mn    & 1,00  & 0,21  & 0,45  & 96 \\
    \bottomrule
    \end{tabular}%
  \label{tab:methodValidation}%
\end{table}%


\subsection{Ergebnisse und Diskussion}
Tabelle \ref{tab:methodValidation} enthält die Parameter zur Validierung der analytischen Methode, wie dem Korrelationskoeffizient, Sensitivität
und der Wiederfindung. Tabelle \ref{tab:refAnalyse} präsentiert die Richtigkeit der Methode mit einem inhouse vermessenem Joghurt, der als Referenzmaterial dient. Tabelle 4 umfasst die ermittelten Konzentration der gesuchten Elemente. 
% Table generated by Excel2LaTeX from sheet 'ICP-OES Ergebnis'
\begin{table}[htbp]
  \centering
  \caption{Vermessung der inhouse Referenz}
    \begin{tabular}{lccc}
    \toprule
    Element & \makecell{Zertifizierter \\ Gehalt [$\frac{mg}{kg}$]}  & \makecell{Gemessene \\ Werte [$\frac{mg}{kg}$]} & WFR [\%] \\
    \midrule
    Ca    & 10200 $\pm$ 400   & 8975 $\pm$ 277 & 88 \\
    K     & 11200 $\pm$ 500   & 10340 $\pm$ 668 & 92 \\
    Mg    & 880   $\pm$ 50    & 763 $\pm$ 21 & 87 \\
    Na    & 2700  $\pm$ 100   & 2581 $\pm$ 109 & 96 \\
    Fe    & 2,8   $\pm$ 0,3   & 2,15 $\pm$ 1,00 & 77 \\
    Mn    & 0,19  $\pm$ 0,01  & < LOD      & - \\
    \bottomrule
    \end{tabular}%
  \label{tab:refAnalyse}%
\end{table}%
