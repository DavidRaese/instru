\section{Einleitung}
Milch und Milchprodukte sind eine gute Quelle für viele der essentiellen Mineralien, die der Mensch über seine Ernährung aufnehmen kann \cite{goIn}. Joghurt ist nicht nur in dieser Hinsicht sehr vielversprechend, sondern ermöglicht auch noch die Zufuhr von Darmbakterien, die die Darmflora verbessern sollen \cite{probiHe}. Da seit den letzten Jahren der gesundheitliche Aspekt von Lebensmitteln immer wichtiger wird, ist dies eine gute Methode Joghurt noch weiter zu veredeln. Durch die verschiedenen Produktionsschritte können die Milchprodukte aber nicht nur für die Gesundheit zuträgliche Mineralien enthalten, sondern auch schädliche, wenn nicht sogar giftige \cite{heMeCon}. Daher ist es von äußerster Wichtigkeit bei Nahrungsmitteln, die in so hohem Maße verzehrt werden (ca. 108kg Milch pro Kopf und Jahr Weltweit \cite{wDSit}),konstant den Gehalt dieser Metalle zu überprüfen. 
 
Der charakteristisch säuereliche Geschmack, sowie die zähe Konsistenz von Joghurt wird hauptsächlich über die teilweise Umwandlung der Laktose zu Milchsäure, durch zugefügte Milchsäurebakterien, erzielt. Die so erzeugte Milchsäure führt zur Denaturierung der äußeren Schutzhülle der Eiweißmoleküle. Angesichts der freigelegten Proteine kann es zur Vernetzung ebendieser kommen, wobei in den Zwischenräumen das in der Milch enthaltende Wasser eingeschlossen wird. 

Um das komplette Geschmacksprofil bestimmen zu können, werden mittels HPLC und GC die
organischen Verbindungen untersucht.

Diese Studie versucht somit einerseits die Frage zu beantworten, wie hoch die Konzentration der Mengen- und Nebenelmenten in Joghurt ist und ob die Grenzwerte \cite{refWerte} eingehalten werden.
Anderseits, welche Verbindungen für das Geschmacksprofil zuständig sind.