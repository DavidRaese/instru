\newpage
\subsection{HPLC}
\subsubsection{Materialien}
Zur Vermessung wurde ein Agilent 1200 System, ausgerüstet mit parallel geschaltetem UV- und RI-Detektor, verwendet. Der UV-Detektor war auf die Wellenlängen 285nm und 220nm eingestellt. \\
Die Analyse wurde isokratische, mit einer Flussrate von 0.8ml/min bei 65°C durchgeführt. Zur Trennung wurde eine 7,8mmx100mm Ionenaustauschsäule (Rezex RFQ-Fast) mit einer 1mM schwefelsaueren mobilen Phase verwendet. 

\subsubsection{Kalibration und Berechnung}
Zur Quantifizierung der vermessen Moleküle wurde ein externe Kalibration angefertigt. Dafür wurden die Elutionszeiten mittels Einzelstandards, im Konzentrationsbereich von 0,5 - 10mM, der Zucker und Säuren hergestellt und vermessen. Zur anschließenden Berechnung der Konzentrationen wurde die Signalfläche, bei ausreichender Auflösung (>1,5), gegen die eingesetzten Konzentrationen aufgetragen. Über die Regressionsgerade konnte dann die jeweilige Konzentration ermittelt werden. 

\subsubsection{Probenvorbereitung}
Referenzprobe (100mg) und  Eigenprobe (500mg) wurden auf 10mL mit deionisierten Wasser verdünnt. Anschließenden kamen die Proben zum Degasen für 15min in ein Ultraschallbad (Elma Transonic T700/H). Zuletzt wurden die Proben noch mittels einem Nylon 66 Spritzenaufsatz (Durchmesser 25mm; Porendurchmesser 0,22µm) gefiltert. 

\subsubsection{Ergebnisse und Diskussion}
Zur Detektion wurden zwei verschiedene Sensoren verwendet. Einerseits einen refraktometrischen Detektor (RI), welcher mittels variierendem Brechungsverhalten verschiedener Substanzen vermessen kann. Diese Methode ist unspezifische, d.h. es kann jede Substanz detektiert werden, aber nur bei hinreichend hohen Konzentrationen. Andererseits wurde eine UV-Detektor verwendet, dieser ist circa 10mal sensitiver als der RI-Detektor, kann aber nur Substanzen vermessen, die UV-Strahlung absorbieren. Von den zu untersuchten Stoffen kann einzig die Milchsäure mittels UV detektiert werden, da sie als einzige gesuchte Substanz eine Doppelbindung besitzt. \\
Betrachtet man das RID Chromatogramm, erkennt man dass Laktose am schnellsten eluiert, da es am Größten (342g/mol) und ungeladen ist. Darauf folgt Galaktose (180g/mol) und zuletzt eluiert Milchsäure (90g/mol). Im Falle der Milchsäure ist der Ladungszustand zu beachten, da dieser bei der Ionenaustauschchromatographie extrem wichtig ist. Ist der pH-Wert der mobilen Phase größer dem pK$_a$-Wert der Milchsäure, liegt diese geladen vor und wird dementsprechend stärker zurückgehalten.


\begin{table}[htbp]
  \centering
  \caption{Ergebnisse der HPLC}
    \begin{tabular}{lllr}
    \toprule
     & \makecell{Eigenprobe \\ mg/L} & \makecell{Referenzprobe \\ mg/L} & R$^2$ \\
    \midrule
    Galactose & $64,7 \pm 1,87$ & $35,4 \pm 1,07$ &  \\
    Lactose & $336 \pm 8.19$ & $140 \pm 4,14$ &   \\
    Milchsäure & $29.9 \pm 1.08$ & $14.2 \pm 0.29$ &  \\
    \bottomrule
    \end{tabular}%
  \label{tab:hplcErg}%
\end{table}%


\subsubsection{Methodenentwicklung}
Für optimale Effizienz sollte die Auflösung 1,5 betragen. Damit ist eine ausreichende Trennung bei geringster Zeit gewährleistet. Verglicht man die Auflösung des HPLC Gerätes (Agilent 1260 infinity) bei verschiedenen Temperaturen mit Hilfe einer Standardlösung (0.1M mit Harnsäure, Laktose, Galaktose und Milchsäure), wird ersichtlich, dass durch die Erhöhung der Temperatur die Auflösung verringert wird. Im Falle der zu untersuchenden Substanzen ist dies ein gewollter Effekt, da sich dadurch die Analysezeit, bei ausreichender Auflösung (>1,5), verkürzt (siehe Abbildung 1). 